%%=============================================================================
%% Methodologie
%%=============================================================================

\chapter{Methodologie}
\label{ch:methodologie}

Tijdens de bachelorproef heb ik volgende stappen ondernomen om de informatie te verzamelen, ik ben begonnen met het bepalen van de onderzoeksvraag. Nadat deze geformuleerd was ben ik begonnen met het verzamelen van de nodige informatie. De vragen die hier naar boven zijn gekomen heb ik aan de hand van een kleine interviews kunnen beantwoorden.

\section{Bepalen van de onderzoeksvraag}

Aangezien ik mijn stage doe bij Combell heb ik aan hun eerst gevraagd of zij een interessant onderwerp hadden in de wereld van de systeembeheerders. Wetende dat ze een van de grootste spelers zijn hier in België op vlak van hosting en 24/7 service moeten ze ook meegroeien met de laatste nieuwe trends. Hun infrastructuur en oplossingen voor de klant is vrij uitgebreid, daarom stelde ze me ook de vraag of ik graag iets wou doen in de systeembeheerwereld of liever iets meer hands on en dan meer gericht op de devops wereld. Ook al studeer ik af in als als systeembeheerder, toch blijf ik programmeren wel nog steeds leuk vinden. Daarom heb ik ook de optie van een opdracht in het thema devops verkozen. Na wat over en weer gemail met de mensen van het operationsteam en van het platformsteam hebben we besloten dat Combell graag een onderzoek had gevoerd naar hoe we containers kunnen integreren in het huidige aanbod. Dit met de extra moeilijkheid dat we niet alleen een simpele virtuele machine opstarten en daar Docker containers opzetten, maar dat daze ook schaalbaar moest zijn en met een zekere vorm van automatisatie. Dit was de aanzet tot de onderzoeksvraag.

\section{Literatuurstudie}

Na het bepalen van de onderzoeksvraag was het tijd om de bestaande infrastructuur in kaart te brengen en om de toekomstige technologieën te gaan onderzoeken. Aangezien ik stage deed als systeembeheerder was mijn eerste stageopdracht een algemeen beeld te vormen van wat Combell aan te bieden heeft en hoe de verschillende technologieën samen werkte. Dit was in het begin zeker geen eenvoudige opdracht aangezien alles zo uitgebreid was. Ik had dus ook een 4 tal weken nodig om me te kunnen inwerken in het hele systeem. Eenmaal ik de gebruikte technologieën verstond en mijn weg wat vond in het hele systeem kon ik beginnen aan het onderzoek. Aangezien de technologieën die ik moest gaan gebruiken vrij nieuw waren, waren er dus ook een aantal bijeenkomsten waar er uitvoerig over gesproken zou worden. Daarom leek het mij geen slecht idee om ook een te gaan luisteren op het Config Management Camp dat elk jaar doorgaat op campus Schoonmeersen van de HoGent. Hier heb ik 2 dagen een aantal talks me gevolgd die de onderwerpen Kubernetes en OpenStack aansneden. Deze gaven mij al het eerste inzicht aan wat ik me kon verwachten tijdens mijn onderzoek.

\section{Interviews}

Na het eerste onderwerp in mijn literatuurstudie namelijk de huidige infrastructuur onderzoeken, kwamen er toch een aantal vragen naar boven. Aangezien ik nog niet vertrouwd was met de technologieën die ze bij Combell hanteren, meerbepaald het OpenStack platform en VMware producten ben ik met mijn vragen bij Maarten Steenhuyse, Expert Devops Engineer, Melissa De Witte, Teamlead Systeembeheer en Wesley Hof, Team Lead Platforms gaan praten. Zij, samen met Tom Temmerman, Operations Engineer Hebben mij een zeer duidelijk antwoord kunnen bieden op de vragen betreffende de gebruikte technologieën. Deze vragen kan u vinden in Bijlage XX. Ook is het netwerk, dat deze infrastructuur nodig heeft een enorme uitdaging. Daarom ben ik ook een aantal dagen op stap geweest met Jonathan Lataire, Data Center Engineer om infrastructuur en de datacenters van dichtbij te bekijken en om een beter beeld te krijgen van hoe alles nu samen werkt. Door de ervaring en kennis van zowel het platforms team als het operations team en onze Data Center Engineer Jonathan  heb ik veel interessante en bruikbare informatie kunnen verzamelen die van groot belang waren voor het verloop van de bachelorproef.


